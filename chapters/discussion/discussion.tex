\chapter{Discussion}
\label{ch:discussion}

The main aim of this thesis was to study the numerical stability of neuroimaging pipelines focusing on the effect
of operating system variability. For this purpose, we leveraged system call interception techniques, including
the ReproZip tool and perturbation models such as Monte-Carlo arithmetic (MCA)~\cite{Parker1997-qq} as an extension of
standard floating-point arithmetic that exploits randomness in basic floating-point operations. 
With this in mind, this chapter first considers the discussion of the findings and contributions.
We then discuss the implication of the results and limitations of the current methodology in this work.
This chapter concludes with recommendations for future studies.


\subsection{Thesis findings}

Throughout this thesis, we demonstrated the role of numerical errors in different computational environments in neuroimaging pipelines.
In Chapter~\ref{ch:c1}, we introduced Spot, a tool to detect the source of numerical differences in pipelines
executed in different operating systems. 
This work localized the origin of processes that hamper bitwise reproducibility as the effect of OS updates.
Among them, we observed registration processes as the highly contributed source of instability in the FSL tool. 
We found that differences can propagate during the pipeline execution and substantially impact
the results of the entire pipeline (e.g., segmentations created by FreeSurfer).
We also captured different results between subjects, showed the sensitivity of the analyses to dataset selection.
Notably, the irreproducible processes identified by Spot were all associated with dynamically linked executables,
showing that library updates do not impact statically linked executables.


The numerical differences were caused by truncation and round-off errors
due to the precision limitations of the floating-point arithmetic.
This is uncontrolled noise that originated from the updates of the system libraries associated with
dynamically linked executables of the pipeline. Thereby, the observed instability across OSes
depends on which operating systems are used. This limits the evaluation of the underlying stability of a particular tool.

Thereafter, in Chapter~\ref{ch:c2}, we presented a framework to model the numerical uncertainty induced by OS updates in a control condition using Monte-Carlo arithmetic. 
We obtained an accurate simulation of OS variability using the FL library in which perturbing mathematical GNU/Linux library (libmath).
% This suggests that numerical perturbations introduced by OS updates might be in the range of machine error (virtual precision of 53 bits).
We found that the pipeline could be implemented exclusively with lower precision of floating-point representations (single-precision)
without loss of results precision. This would substantially decrease the pipeline memory footprint and computational time.
However, the finding showed a very low number of significant bits, 5 out of 15 available bits.
This motivates further investigation of the numerical stability of the main components of the tested pipelines, such as linear and non-linear registrations.
Also, it is notable that OS- and FL-induced variability were on a similar order of magnitude as subject-level effects,
showing that we applied a reasonable amount of perturbations.

This framework works on the level of the shared libraries, so there is no need for recompilation or modification
of the pipeline or any other sources.
We, therefore, believe that pipelines that depend
on different third-party mathematical libraries could be studied similarly by building fuzzy
versions of these dependencies. This motivated us to investigate the numerical quality in
more applications like tool variability using the proposed MCA-based method.

Finally, in Chapter~\ref{ch:c3}, we investigated the type of variability introduced in variations between software packages.
We compared numerical variability with tool variability across three of the most popular software packages in neuroimaging.
% This work presents a comparison of numerical and software variability for group-level functional MRI analysis
We found significant numerical variability that was comparable to tool variability in some brain regions.
We also found that numerical instability in individual analyses was attenuated in group analyses.
It is notable that we obtained more uncertainty on thresholded results than unthresholded maps, probably due to
different thresholds used in different tools.
% This raises further investigations on thresholding techniques to reduce the associated instabilities.


\subsection{The implication of results}

This study was a contribution to uncertainty quantification in medical imaging.
The numerical error is often neglected or only partially studied due to the associated engineering challenges among
the various sources of uncertainty involved in medical imaging results, including population selection,
scanning devices and sequence parameters, acquisition noise, image reconstruction algorithms, and methodological flexibility.
Our work proposed the methodologies and implementations to address this issue.

The current approach to address numerical instability resulting from OS
updates is mainly to ignore the issue and sweep it under the rug of %static build of programs, or 
Docker containers or other types of virtualization. 
Although building static program and containerization techniques improve
reproducibility across OSes, but small differences remained.
It remains that computational results should be understood as realizations of a random variable resulting
from floating-point arithmetic. The presented techniques in this thesis enable estimating
result distributions, the initial step toward making analyses reproducible across
execution environments, including HPC systems, GPU accelerators, or merely
different workstations.

Results across my thesis showed the significance of numerical instabilities in neuroimaging pipelines
and demonstrated numerical analysis techniques such as MCA as valuable methods for evaluating the associated variability. 
Moreover, in related work~\cite{kiar2020numerical}, it has been suggested that capturing this variability may improve the robustness of scientific
findings. This finding importantly highlights how numerical variability may be regarded as a feature that should be
taken into considerations by the pipeline developers in the neuroimaging and other scientific domains.

We demonstrated the numerical noise sensitivity of two of the most complex pipelines in neuroimaging,
HCP preprocessing pipelines, PreFreeSurfer and FreeSurfer. Technically, these pipelines consist of a mix of tools
assembled from different toolboxes through a variety of scripts written in different languages. In addition to the preprocessing steps,
we evaluated the numerical stability of a complete fMRI analysis using three of the most popular tools in neuroimaging, including FSL, SPM, and AFNI.
We, therefore, believe that the results presented in this thesis would apply to a wide range of other pipelines. 
However, the current methodology is limited to Linux operating systems. 
Our findings are likely to generalize to OS/X or MS Windows, although future work would be needed to confirm that.

Moreover, the interposition technique is only applicable to intercept system calls in dynamically linked programs. 
Further investigations are needed if we aim to evaluate the stability of the pipelines with statically linked executables.


\subsection{Recommendations for the future research}

Among the processes that contribute to the pipeline instability, some of them substantially
have a higher effect on results. For instance, iterative computations can accumulate rounding errors in
some cases and significantly amplify errors. Therefore, evaluating the stability of pipeline components as
well as the entire pipeline helps to identify processes that propagate and amplify errors within pipelines and
then substitute them for more stable tools that perform a similar function, ultimately improving the quality
of the pipeline.
It is important to propose numerical debugging tools to identify such numerical bugs raised during the
execution of a complicated pipeline. This is a well-studied problem in the floating-point error characterization
research field, which has resulted in the development of debugging tools such as VeriTracer~\cite{chatelain2018veritracer},
FpDebug~\cite{benz2012dynamic}, Verrou~\cite{fevotte2019debugging}.
These tools automatically replace all the floating-point operations with their MCA counterparts
or other stochastic arithmetics to evaluate the numerical quality of the computation and pinpoint the parts
of the source code. They instrument pipelines at compilation time in the compiler-based tools that need to
access the source code, or running time in the Valgrind-based tools that add computation time overhead.
Moreover, some of these tools need pre-knowledge of the functions or variables to be traced in the pipeline.

Using a combination of techniques developed in this thesis, in future work,
we can create a debugging tool at the level of system call processes without
recompiling the source code to identify the processes with the highest impact on results in the pipeline due
to the operating system variability. 
This functionality could leverage the tool
developed in Chapter~\ref{ch:c1} and the interposition technique to inject MCA perturbations described in Chapter~\ref{ch:c2}.
Moreover, we can use the principle of minimization by delta-debugging to search through a large number of processes. 
The delta-debugging algorithm is a general technique that can automatically narrow down the failure caused by changing circumstances
that are critical to producing the bug, such as program input, program code, environmental configurations,
etc~\cite{zeller1999yesterday,zeller2002isolating}. Instead of working on the program's code, we can apply the delta-debugging methodology to the
program history by comparing various versions until the faulty change is found. This enables to identify
of those processes in the program that significantly have higher effects on results. Finding these processes could
narrow down further investigations toward stabilizing the pipelines.

Improving the instability needs actions different from evaluation and localization of instability.
It is not a general approach to stabilize the entire pipeline; we should look at the relevant code section
to find an appropriate solution. For example, we found that linear and non-linear registration processes introduce errors
in both FreeSurfer and PreFreeSurfer pipelines. On the other hand, we know that the registration procedure is sensitive to
the initialization of the optimization method used~\cite{Glatard2018hbm}. 
A possible method to address such instabilities is using the bootstrapping technique.
In~\cite{Glatard2018hbm}, the authors explained that bootstrapping is an efficient technique to improve
the robustness of motion estimation. The bootstrap version of the pipelines computed the median transformation
results from the 30 samples. In addition to bootstrapping, it is shown that the bagging technique can
reduce the effect of the medians of the parameters of the 30 transformations. Bagging, also called bootstrap
aggregating, is a simple and powerful ensemble method. It helps reduce both bias and variance in the results.
So, we can possibly stabilize pipelines and improve their accuracy using aggregates of results obtained with
data perturbations. However, it is a compute-intensive technique that should be used only when no other
solution to the instability is available. 
Further study needs to be conducted into the processes involved in analysis instability.

The main goal of this thesis was to evaluate instabilities within neuroimaging pipelines. 
An exciting research topic for future work could be finding that the variability arising from numerical perturbations may contain
meaningful signals. This suggests that the perturbation model is not only helpful as a method for measuring
the stability of analyses but that it can be applied to improve their quality. 
For this purpose, we can also apply other types of perturbations.
Given that FL only perturbs basic mathematical functions, we expect more numerical variability
by perturbing operations that rely on the linear algebra libraries BLAS and LAPACK.
For future study, this can be evaluated using MCA-instrumented
versions of BLAS and LAPACK along with other libraries available in the Fuzzy project in Verificarlo’s
GitHub repository at \url{github.com/verificarlo/fuzzy}.

