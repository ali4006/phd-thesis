\chapter{Introduction}
\label{ch:intro}

Reproducibility is regarded as a fundamental concept in the scientific 
community. Research findings are expected to be reproducible so that 
their authenticity and reliability can be evaluated. The goal of my 
research is to investigate the reproducibility of computational analyses across different 
computing environments. In particular, we are mostly interested in 
neuroimaging as a case study. We aim to present techniques to evaluate 
the numerical instability of analysis across different computing environments 
instead of masking the reproducibility problem by fixing parameters.

In this chapter, we summarize the main definitions and 
principles relevant to reproducibility. We describe the context of the 
current ``reproducibility crisis" acknowledged in several scientific 
disciplines. Multiple studies have shown that research findings could 
not be reproduced by independent researchers, or even by the original 
researchers themselves. We discuss the main causes for this lack of 
reproducibility, focusing on the computational aspects. 
Additionally, we describe different kind of analysis of 
neuroimaging data and their implemented software which are considered 
through this thesis. 


\section{Reproducibility Definitions}

There are different definitions for the terms reproducibility, 
repeatability, and replicability, which leads to confusion since the 
same words are used for different concepts. Here we present different 
terminologies found in the literature and summarized 
in~\cite{plesser2018reproducibility} (see 
Table~\ref{table:definitions}).
 
According to Peng's definition~\cite{peng2011reproducible}, 
reproducibility is defined as the ability to regenerate the same 
results as the original findings when the experiment is reanalyzed 
given exactly the same analytic methods and data. Reproducibility 
ensures that independent scientists can reproduce the same results 
using the same data and procedure as published in the original 
publication. On the contrary, replicability is defined as the ability 
to obtain similar results as published in the original study when 
the experiment is reimplemented using independent data and analytic 
methods. Replicability confirms scientific claims and ensures that 
independent investigators can produce consistent results, using new 
data and methods. Peng introduced the idea of reproducibility 
spectrum based on his definition of reproducibility, which defines a 
minimum standard to evaluate the authenticity of scientific claims. 
In this spectrum, according to what data and sources are available, a 
full replication or no replication of a study can be achieved. 
The same definitions of reproducibility and replicability 
are also used by Schwab et al.~\cite{schwab2000making}.

Donoho et al.~\cite{donoho2009reproducible} defined reproducible 
computational research as a process where ``all details of 
computations such as code and data are made conveniently available to 
others". The authors associate reproducible research with open science,
including open code and data. They observe that 
reproducibility can be achieved by publishing the experimental 
resources over the Internet, which facilitates versioning, testing, 
discovery and access to the research materials. 

In addition, Goodman et al.~\cite{goodman2016does} renamed Peng's 
reproducibility and replicability as methods 
reproducibility and results reproducibility respectively, and adopted a 
new terminology called inferential reproducibility.
From Goodman's terminology, exactly the same data and procedure are 
reanalyzed in methods reproducibility. Result reproducibility is 
equivalent to Peng's replicability terminology which is defined as 
getting almost the same results compared to the original study from an 
independent replication of a study. Also, inferential reproducibility 
is defined as getting the same conclusions from either a reanalysis of 
the original study or an independent replication of a study with 
different data and analysis procedures.
 
Furthermore, the Association for Computing Machinery 
(ACM)~\cite{acm2016terminology} proposes three different categories of 
repeatability, replicability, and reproducibility. Repeatability is 
defined as repeating computation by the same experimental setup 
including operator team, operating conditions, location, and measuring 
system. 
Similar to replicability, repeatability uses identical experimental 
conditions except performer team.
This means that an independent group can achieve the 
same results through the same experimental parameters. 
Additionally, reproducibility is measured by performing 
different experimental setups via different teams 
independently. It should be noted that reproducibility and 
replicability are used inversely compared to Peng's definitions. 

\setlength{\tabcolsep}{1pt}
\begin{table}[H]
\hspace*{-1cm}
\caption{Overview of definitions.}
\label{table:definitions}
\centering
\begin{threeparttable}
\begin{tabular}{@{}llllll@{}}
\toprule
Schwab et al.$_{(2000)}$ & Donoho et al.$_{(2009)}$ & Peng$_{(2011)}$ & 
ACM$_{(2016)}$ & Goodman et al.$_{(2016)}$ \\ \midrule
\makecell{Reproducibility \\ Replicability} &   \makecell{Open \\ code 
and data} &   \makecell{Reproducibility \\ spectrum}  &   
\makecell{Repeatability \\ Replicability \\ Reproducibility} & 
\makecell{Method reproducibility \\ Results reproducibility \\ Inferential reproducibility}    \\ \bottomrule
\end{tabular}
\end{threeparttable}
\end{table}

In addition, numerical reproducibility is defined as the ability to 
regenerate bit for bit identical results from multiple 
runs~\cite{hill2017numerical}. Two files will be considered numerically 
reproducible if they have identical binary contents. Binary 
comparison is calculated by comparing checksums. It must be pointed out 
that a computation might be reproducible based on Peng's definition, 
but not be numerically reproducible. For instance, small numerical 
errors created during the pipeline execution may hamper numerical 
reproducibility, but be negligible in the final results.

Reproducibility, as the cornerstone of scientific research, guarantees 
the reliability of results. A reproducible study provides a context in which one can get results consistent with the original work.
In addition, it not only saves a great deal of time, but also 
enables others to use existing works as a part of their 
experiments~\cite{plesser2018reproducibility}. 
In our work, we follow Peng's definition of reproducibility 
unless we directly refer to numerical reproducibility. 
We seek to identify why such reproducibility may not be ensured, focusing particularly on computational aspects.

\section{Reproducibility Crisis}

Recently, scientists began to realize that the results of many 
scientific experiments were neither replicable nor reproducible. This 
realization is termed the reproducibility crisis. In this section, 
we provide an overview of evidence for the
reproducibility crisis, which has raised important concerns in the 
scientific community.

Ioannidis~\cite{ioannidis2005most} introduces an important framework to 
demonstrate the probability that research findings are false, and the 
propagation of valid findings in a given research field. He defined 
biased research as ``the combination of various design, data, analysis, 
and presentation factors that tend to produce research findings when 
they should not be produced". Consequently, biased research, focused on 
an individual discovery rather than on broader evidences, decreases the 
chance of true findings. He concluded that ``most of the research 
claims are less likely to be true than false for most fields and 
research designs". The author argued that the probability of true 
findings is highly dependent on the number of similar studies in a 
scientific field, the number of researchers/teams involved in the study, 
and the flexibility of analytic models, definitions, and outcomes. For 
example, the smaller the studies conducted in a scientific field, the 
less likely the research findings are to be true. 

To highlight the importance of scientific reproducibility, the survey 
in~\cite{baker20161} collected data from 1500 scientists among 
different disciplines mostly from  biology, medicine, and engineering. 
This report found that 70\% of the scientists polled could not 
replicate another scientist's findings, and even 50\% failed to 
reproduce their own results. Moreover, this study listed some of 
the main reasons that lead to irreproducibility of analysis 
such as poor statistics, the pressure to publish 
and then selective analysis. 
With this, over 50\% of the scientists polled believed that there was a 
significant crisis.

Furthermore, some studies underline the reproducibility issues of 
current analysis methods in neuroimaging~\cite{jovicich2009mri, 
muller2017altered, eklund2016cluster}. For example, to evaluate 
reproducibility of a group of functional MRI (fMRI) analyses, the study 
in~\cite{eklund2016cluster} collected resting-state fMRI data from 499 
healthy controls. Using this dataset, they found that the most common 
software packages for fMRI analysis (SPM, FSL, AFNI) can result in a 
high degree of false positives, up to 70\% compared with the expected 
5\%. These results question the validity of some 40,000 fMRI studies 
and may have a large impact on the interpretation of neuroimaging 
results. All these evidences show a significant crisis in 
reproducibility of experiments that should be taken into consideration in 
scientific communities.

\section{Main Computational Causes of Irreproducibility} 

The main barrier to reproducibility in many cases is 
that the analysis program, data, and analytic methods are no longer 
available. Addressing this problem requires the development of a culture of 
reproducibility among scientific community, which enables the third 
party to reproduce the same experiment~\cite{peng2011reproducible, 
stodden2016enhancing}. 

From the computational point of view, reasons such as the lack of 
details of the computational environments can contribute to 
irreproducibility of research results. Analyses need sufficient 
information on code, software, hardware, and implementation details to 
be computationally reproducible. In addition, capturing such 
information is complicated, particularly in domains where results rely 
on a sequence of complex analyses such as neuroimaging pipelines. 
To overcome this complexity, a mechanism called provenance capturing is
designed to encompass all dependency information of the computational 
analysis such as input/output data, processing steps, and 
detail of computing environments. 

Furthermore, the variety of computational infrastructures including 
workstation types, parallelization methods, operating systems, and 
analysis packages are known to influence reproducibility because of the 
creation of small numerical errors~\cite{Gronenschild2012, 
diethelm2012limits, Glatard2015}. 
For instance, we will explain further that different order of 
summation operation of floating-point numbers can lead to creation of 
small numerical differences. 
The propagation and amplification of these tiny 
errors by analysis pipelines may cause reproducibility issues. We 
will discuss in more details the effect of each one of these factors in Chapter~\ref{ch:background}.

\section{Analyzing Neuroimaging Data}

There are many different kinds of imaging techniques to acquire 
brain image data. The most common techniques are structural magnetic 
resonance imaging (sMRI), functional magnetic resonance imaging (fMRI) 
and diffusion magnetic resonance imaging (dMRI).

Structural neuroimaging deals with the anatomical structure of the brain and helps
diagnose brain injury and certain diseases such as tumor and stroke.
The main software packages used for sMRI are 
CIVET~\cite{ad2006civet}, FreeSurfer~\cite{fischl2012freesurfer}, and 
FSL (FMRIB Software Library)~\cite{jenkinson2012fsl}. 
As opposed to structural imaging, functional imaging is used to measure 
brain function based on 
specific tasks completed by subjects such as listening to sounds, 
reading, or small movements. 
Functional imaging identifies the areas of 
the brain that are involved with responding to the tasks. 
In addition to task-based fMRI, an explicit task may not be performed 
to identify the functional activity of brain in a resting-state 
condition (RS-fMRI). The main 
software packages that implement fMRI processing are SPM (Statistical 
Parametric Mapping)~\cite{spm}, FSL(FMRIB Software Library)~\cite{jenkinson2012fsl},
and AFNI (Analysis of Functional 
NeuroImages)~\cite{cox1996afni}. Diffusion imaging is another kind of 
MRI analysis that measures the anatomical connectivity between 
regions, and the main toolboxes are DIPY (Diffusion Imaging in 
Python)~\cite{garyfallidis2014dipy}, MRtrix~\cite{tournier2012mrtrix}, 
and FSL.

Depending on the analysis type, several steps can be involved in a 
neuroimaging study. Generally, the analysis procedure can be divided into 
pre-processing and statistical steps. Pre-processing steps are taken 
to prepare data for the statistical analysis and are common between all 
analysis modalities, including brain extraction to separate the brain 
tissues from the other parts, or brain alignment which aligns a brain 
extracted image with a reference image such as the one produced by 
MNI (Montreal Neuroimaging Institute)~\cite{evans1992anatomical}. 
After pre-processing steps, depending on the
modality of analyses (e.g., sMRI, fMRI, and dMRI), statistical analyses 
are applied to make inferences.

The various pre-processing and analysis steps involved in a 
neuroimaging experiment are often combined in programs called workflows 
or pipelines. Pipelines are used to automate data analysis and accelerate 
the processing of complicated analyses. 


\section{Thesis Outline}

This thesis aims to study the numerical stability of neuroimaging pipelines focusing on the effect
of operating system variability. For this purpose, we leverage system call interception techniques including the
ReproZip tool, and perturbation models such as Monte-Carlo Arithmetic (MCA)~\cite{Parker1997-qq} as an extension of
standard floating-point arithmetic that exploits randomness in basic floating-point operations. 
This thesis is manuscript-based, meaning that each of the three body chapters, as listed below,
is an exact copy of either a published or aim to publish manuscript.
% The major contributions of my thesis are listed below as the separate chapters that we published or aim to publish.

C.I – File-based localization of numerical perturbations in data analysis pipelines (Chapter~\ref{ch:c1})

C.II – Accurate simulation of operating system updates in neuroimaging using Monte-Carlo arithmetic (Chapter~\ref{ch:c2})

C.III – Comparing software variability across and within fMRI analysis packages (Chapter~\ref{ch:c3})


In Chapter~\ref{ch:background}, we will review the background material related to this thesis in general.
Chapter~\ref{ch:c1} will introduce Spot, a tool to detect the source of numerical differences in complex pipelines
executed in different operating systems. This chapter is completed and published in the
GigaScience journal. Chapter~\ref{ch:c2} will then study whether the MCA method is a good perturbation
model for evaluating pipeline stability across operating systems. This chapter is also completed and published
in the MICCAI workshop on Uncertainty for Safe Utilization of Machine Learning in Medical Imaging (UNSURE).
Chapter~\ref{ch:c3} will present a comparison of numerical and software variability through Monte-Carlo arithmetic.
This chapter will be submitted to the Human Brain Mapping (HBM) journal by the end of Winter 2022.
The thesis will then follow with a discussion and a conclusion in Chapters~\ref{ch:discussion} and ~\ref{ch:conclusion}, respectively.

