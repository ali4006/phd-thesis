\chapter{Conclusion}
\label{ch:conclusion}

The numerical stability of the computational analyses plays an important role in the reproducibility
of the scientific findings. It has been evaluated that results are sensitive to the
computing environment changes such as operating system and analysis toolbox, particularly
in computationally intensive domains where results rely on a series of complex computations.
Throughout this thesis, we demonstrated the impact of numerical instabilities on neuroimaging results. 
We implemented Spot tool that localizes irreproducible processes between operating system variations.
Moreover, we presented an MCA-based method to apply numerical perturbations in floating-point operations,
simulating OS variability and studying the numerical instabilities more comprehensively.
This was done using the Linux interception utility LD\_PRELOAD,
which transparently interposed system calls to an instrumented counterpart.
We expanded our findings by capturing numerical variability among different software packages and comparing it with the tool variability.
As the successful completion of this thesis, we built the tools that enable software developers
and researchers to evaluate the stability of their pipelines and results when dynamically
linked mathematical libraries are changed. Furthermore, the findings of this thesis could be
used to narrow down further investigations toward stabilizing pipelines.
