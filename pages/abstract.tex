\begin{abstract}


The changes of computational infrastructure, including operating system, software version, and hardware
architecture, introduce variability in neuroimaging analyses that could affect the reproducibility of the
scientific conclusions. This is probably due to the creation, propagation, and amplification of numerical
instabilities in analysis pipelines. In this regard, it is critical to identify numerical instabilities to make
experiments computationally reproducible.
In this thesis, we characterize the numerical stability of commonly-used complex pipelines in the
context of neuroimaging analysis across the operating systems and provide accessible tools for developers and
researchers to evaluate their pipelines and findings. First, we present Spot tool that identifies the processes
from which differences originate and the path along which they propagate in the pipelines. In the next step,
to study the numerical instabilities more comprehensively, we introduce controlled numerical perturbations
to the floating-point computations using the Monte-Carlo Arithmetic method. We propose an interposition
technique to model the effect of operating system updates on analysis pipelines using the Monte-Carlo
arithmetic. Finally, leveraging the interposition technique, we compare numerical variability with tool variability
in an fMRI analysis. All the methods implemented in this thesis can be used to facilitate further
investigations toward stabilizing pipelines.


\end{abstract}